\documentclass[letterpaper,12pt]{article}
\usepackage{amsmath}
\usepackage{graphicx}
\usepackage[margin=1in,letterpaper]{geometry}
\usepackage{siunitx}
\usepackage[numbers]{natbib}
\usepackage{hyperref} % 可点击的引用和URL
\usepackage{amsfonts} % 更多数学符号
\usepackage{booktabs} % 更好的表格格式
\usepackage{caption} % 自定义图表标题
\usepackage{listings} % 代码高亮显示
\usepackage{xcolor} % 彩色文本和代码块
\usepackage{algorithm} % 算法环境
\usepackage{algorithmic} % 算法描述
\usepackage{enumitem} % 更好的列表格式
\usepackage{titling} % 用于自定义标题页
\usepackage{fancyhdr} % 页眉页脚设置

% 支持中文显示
\usepackage{xeCJK}
\setCJKmainfont{STHeiti} % 使用系统上已安装的中文字体

% 设置页面样式
\pagestyle{fancy}
\fancyhf{}
\renewcommand{\headrulewidth}{0pt}

% 重新定义maketitle
\pretitle{%
    \begin{center}
    \LARGE
    \vspace*{2cm}
}
\posttitle{\end{center}}

\preauthor{%
    \begin{center}
    \vspace{3cm}
}
\postauthor{\end{center}}

% 设置代码高亮样式
\definecolor{codegreen}{rgb}{0,0.6,0}
\definecolor{codegray}{rgb}{0.5,0.5,0.5}
\definecolor{codepurple}{rgb}{0.58,0,0.82}
\definecolor{backcolour}{rgb}{0.95,0.95,0.95}

\lstdefinestyle{mystyle}{
    backgroundcolor=\color{backcolour},   
    commentstyle=\color{codegreen},
    keywordstyle=\color{magenta},
    numberstyle=\tiny\color{codegray},
    stringstyle=\color{codepurple},
    basicstyle=\ttfamily\footnotesize,
    breakatwhitespace=false,         
    breaklines=true,                 
    captionpos=b,                    
    keepspaces=true,                 
    numbers=left,                    
    numbersep=5pt,                  
    showspaces=false,                
    showstringspaces=false,
    showtabs=false,                  
    tabsize=2
}

\lstset{style=mystyle}

\title{
    \textbf{课程作业一}
}

\author{
    \Large{课程:XXXX} \\[0.5cm]
    \Large{姓名:Lying} \\[0.3cm]
    \Large{学号:XXXXXXXXX} \\[0.3cm]
    \Large{专业:XXXX} \\[0.3cm]
    \Large{学院:XXXX}
}

\date{\vspace{2cm}\today}

\begin{document}

% 生成封面
\begin{titlepage}
\maketitle
\thispagestyle{empty}
\end{titlepage}

\newpage
\setcounter{page}{1}

\begin{abstract}
% 简要总结实验的目的、主要发现和结论。
\end{abstract}

\section{问题描述}
% 描述实验问题及要求

\section{算法原理}
% 详细解释算法的基本原理和理论基础

\section{代码实现}
% 包括代码和解释

\subsection{主要函数与方法}
% 代码示例
\begin{lstlisting}[language=Python, caption=示例代码]
def example_function(param1, param2):
    """
    函数功能说明
    """
    # 初始化
    result = 0
    
    # 主要逻辑
    for i in range(param1):
        result += param2
        
    return result
\end{lstlisting}

\subsection{关键步骤解释}
% 解释代码中的关键步骤和实现思路
\begin{enumerate}[label=\arabic*., leftmargin=*]
    \item 步骤一:...
    \item 步骤二:...
    \item 步骤三:...
\end{enumerate}

\section{实验结果}
% 展示实验结果,可以包括表格、图表等

\subsection{数据分析}
% 分析实验数据

\begin{table}[h]
    \centering
    \caption{实验数据表}
    \begin{tabular}{@{}ccc@{}}
    \toprule
    参数 & 值 & 单位 \\
    \midrule
    参数1 & 值1 & 单位1 \\
    参数2 & 值2 & 单位2 \\
    参数3 & 值3 & 单位3 \\
    \bottomrule
    \end{tabular}
\end{table}

\subsection{可视化结果}
% 插入图表示例
\begin{figure}[h]
    \centering
    % \includegraphics[width=0.8\textwidth]{example_figure.png}
    \caption{实验结果图}
    \label{fig:result}
\end{figure}

\section{讨论与分析}
% 对实验结果进行讨论和分析,包括:
\begin{itemize}
    \item 算法性能分析
    \item 实验结果与理论预期的对比
    \item 可能的改进方向
\end{itemize}

\section{结论}
% 总结实验的主要发现和结论

\appendix % 可选部分,用于补充信息
\section{附录:完整代码}
% 完整代码列表
\begin{lstlisting}[language=Python, caption=完整代码]
# 导入必要的库
import numpy as np
import matplotlib.pyplot as plt

# 完整的代码实现
def main():
    # 代码实现
    pass

if __name__ == "__main__":
    main()
\end{lstlisting}

\end{document}